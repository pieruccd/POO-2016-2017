% book example for classicthesis.sty
\documentclass[
  % Replace twoside with oneside if you are printing your thesis on a single side
  % of the paper, or for viewing on screen.
  %oneside,
  oneside,
  11pt, a4paper,
  footinclude=true,
  headinclude=true,
  cleardoublepage=empty
]{scrbook}
%\usepackage{array}
%\usepackage{lipsum}
\usepackage[linedheaders,parts,pdfspacing]{classicthesis}
\usepackage{amsmath}
\usepackage{amsthm}
\usepackage{hyperref}
\usepackage{graphicx}
\usepackage{listings}

\usepackage{array}

\usepackage[utf8]{inputenc}
\usepackage[T1]{fontenc}
\usepackage{eurosym}
\newcommand{\intervalle}[4]{\mathopen{#1}#2
 \mathclose{}\mathpunct{};#3
 \mathclose{#4}}
\newcommand{\intervalleff}[2]{\intervalle{[}{#1}{#2}{]}}
\newcommand{\intervalleof}[2]{\intervalle{]}{#1}{#2}{]}}
\newcommand{\intervallefo}[2]{\intervalle{[}{#1}{#2}{[}}
\newcommand{\intervalleoo}[2]{\intervalle{]}{#1}{#2}{[}}

\title{Simulation d’une équipe de robots pompiers - TP en temps Libre}
\author{Salah-Eddine Bariol Alaoui\\Majd Fariat\\Dimitri Pierucci}

\begin{document}

\maketitle

%\include{FrontBackMatter/abstract}
%\include{FrontBackMatter/dedication}
%\include{FrontBackMatter/acknowledgements}
%\include{FrontBackMatter/declaration}
%\include{FrontBackMatter/contents}

%\part{Introduction}

\chapter{Première partie : les données du problème}

La première partie du TP, correspond en la réalisation des classes Java représentant le problème et du simulateur. \\

Nous avons séparé les entités en trois packages. \\

\begin{tabular}{|c|c|}
	\hline Package & Robot \\
	\hline \hline 
	classes & Drone \\
			& Robot \\
			& RobotAChenilles \\
			& RobotAPattes \\
			& RobotAroues \\
			& RobotReservoir \\
			& RobotTerrestre \\
	\hline \hline
	Package & Géographie \\
	\hline \hline
	classes & Carte \\
			& Case \\
			& EnumDirection \\
			& EnumNatureTerrain \\
			& Incendie \\
	\hline \hline
	Package & Simulation \\
	\hline \hline
	classes & DonneesSimulation \\
			& Simulateur \\
	\hline \hline
	Package & io \\
	\hline \hline
	classes & LecteurDonnees \\
	\hline
\end{tabular} \\

\section{Package Robot}
Le package Robot contient la classe abstraite Robot ainsi que toutes les classes l'implémentant. \\ 

Nous avons cherché à optimiser l'implémentation des sous-classes de Robot en factorisant le plus possible. \\ 

Les classes Drone, RobotAChenilles et RobotAroues ont la particularité d'avoir un réservoir contrairement à RobotAPattes. Nous les avons donc leurs points communs dans une classe abstraire RobotReservoir fille de Robot. \\

RobotAChenilles et RobotAroues sont différend de Drone, Ils ont une diminution de la vitesse en rapport à la Case où ils se situent et ne vident pas leur réservoir de la même façon de Drone. Nous avons regroupé leurs points communs dans une sous-classe abstraire RobotTerrestre fille de RobotReservoir. \\

Le diagramme UML correspondant du package Robot est le suivant \\
\end{document}




























